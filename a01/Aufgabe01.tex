%%%%%%%%%%%%%%%%%%%%%%%%%%%%%%%%%%%%%%%%%%%%%%%%%%%%%%%%%%%%%%%%%%%%%%%%%%%%%%%%
%% LaTeX Vorlage: ergebnisblatt_vorlage.tex                                   %%
%% Dies ist eine Vorlage fuer die Ergebissblaetter zu den Praktikumsaufagben  %%
%% der Vorlesung 'Einfuehrung in das Wissenschaftliche Rechnen'               %%
%%                                                                            %%
%% Version 2020-04-19, F. Castelli (IANM2, KIT)                               %%
%%%%%%%%%%%%%%%%%%%%%%%%%%%%%%%%%%%%%%%%%%%%%%%%%%%%%%%%%%%%%%%%%%%%%%%%%%%%%%%%
\documentclass[11pt,a4paper]{article}


%% Pakete
\usepackage[ngerman]{babel}
\usepackage[T1]{fontenc}
% \usepackage[utf8]{inputenc}   % Unix
\usepackage[latin1]{inputenc} % Windows
\usepackage[pdftex]{graphicx}
\usepackage{epstopdf}
\usepackage{amsmath,amssymb}


%% Seitenlayout
\usepackage[DIV=12]{typearea}
\setlength{\parindent}{0em}


%% Font Helvetica
\renewcommand{\rmdefault}{phv}


%% Titelinformationen
\title{Einf\"uhrung in das Wissenschaftliche Rechnen\\
  Praktikumsblatt 1\\
  Aufgabe 1 (...)}
\author{Vorname Nachname}
\date{TT.MM.JJJJ}



%%%%%%%%%%%%%%%%%%%%%%%%%%%%%%%%%%%%%%%%%%%%%%%%%%%%%%%%%%%%%%%%%%%%%%%%%%%%%%%%
\begin{document}
  
  %% Titel
  \maketitle
  
  %%%%%%%%%%%%%%%%%%%%%%%%%%%%%%%%%%%%%%%%%%%%%%%%%%%%%%%%%%%%%%%%%%%%%%%%%%%%%%
  \section*{Problemstellung}
  %%%%%%%%%%%%%%%%%%%%%%%%%%%%%%%%%%%%%%%%%%%%%%%%%%%%%%%%%%%%%%%%%%%%%%%%%%%%%%
  Beschreiben Sie hier das Problem.
  
  Text $a$,
  \begin{align*}
    a = b.
  \end{align*}
  
  Zur verbesserten Kompatibilit\"at zwischen Betriebssystemen und
  Zeichen-Kodierungen verwenden Sie f\"ur deutsche Sonderzeichen stets
  \verb+\"A+, \verb+\"a+, \verb+\"O+, \verb+\"o+, \verb+\"u+, \verb+\"u+
  sowie \verb+\ss{}+.
  
  Sollten Sie dennoch Probleme mit der Kodierung von deutschen Sonderzeichen
  haben, schauen Sie nach, welche Einstellung Ihr Editor verwendet und passen
  Sie die Option des 'inputenc' Pakets entsprechend an.
  
  Standardm\"a\ss{}ig ist in Editoren unter Windows 'latin1' (ISO 8859-1) und
  unter Unix (Linux, Mac) 'utf8' eingestellt.
  
  
  
  %%%%%%%%%%%%%%%%%%%%%%%%%%%%%%%%%%%%%%%%%%%%%%%%%%%%%%%%%%%%%%%%%%%%%%%%%%%%%%
  \section*{Ergebnis}
  %%%%%%%%%%%%%%%%%%%%%%%%%%%%%%%%%%%%%%%%%%%%%%%%%%%%%%%%%%%%%%%%%%%%%%%%%%%%%%
  
  Beschreiben Sie hier das Ergebnis.
  
  %%----------------------------------------------------------------------------
  %% Fuegen Sie hier einen Bild wie folgt ein:
  %% - Speichern Sie das gewuenschte Bild im Ordner dieser TeX-Datei
  %% - Entkommentieren Sie den Befehl \includegraphics[...]{...}
  %% - Ersetzen Sie 'Bild_Dateiname' durch den Dateinamen des Bildes
  %%----------------------------------------------------------------------------
  \begin{figure}[ht]
    \centering
    % \includegraphics[width=0.5\textwidth]{Bild_Dateiname}
    \caption{Bildunterschrift}
  \end{figure}
  
  Text.
  
\end{document}
