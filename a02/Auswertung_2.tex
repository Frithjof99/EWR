%%%%%%%%%%%%%%%%%%%%%%%%%%%%%%%%%%%%%%%%%%%%%%%%%%%%%%%%%%%%%%%%%%%%%%%%%%%%%%%%
%% LaTeX Vorlage: ergebnisblatt_vorlage.tex                                   %%
%% Dies ist eine Vorlage fuer die Ergebissblaetter zu den Praktikumsaufagben  %%
%% der Vorlesung 'Einfuehrung in das Wissenschaftliche Rechnen'               %%
%%                                                                            %%
%% Version 2020-04-19, F. Castelli (IANM2, KIT)                               %%
%%%%%%%%%%%%%%%%%%%%%%%%%%%%%%%%%%%%%%%%%%%%%%%%%%%%%%%%%%%%%%%%%%%%%%%%%%%%%%%%
\documentclass[11pt,a4paper]{article}


%% Pakete
\usepackage[ngerman]{babel}
\usepackage[T1]{fontenc}
% \usepackage[utf8]{inputenc}   % Unix
\usepackage[latin1]{inputenc} % Windows
\usepackage[pdftex]{graphicx}
\usepackage{epstopdf}
\usepackage{amsmath,amssymb}


%% Seitenlayout
\usepackage[DIV=12]{typearea}
\setlength{\parindent}{0em}


%% Font Helvetica
\renewcommand{\rmdefault}{phv}


%% Titelinformationen
\title{Einf\"uhrung in das Wissenschaftliche Rechnen\\
  Praktikumsblatt 2\\
  Aufgabe 2 (Konvergenzordnung)}
\author{Lena Hilpp Matr.Nr.: 1941997\\Jan Frithjof Fleischhammer Matr.Nr.: 2115491}
\date{06.05.2020}



%%%%%%%%%%%%%%%%%%%%%%%%%%%%%%%%%%%%%%%%%%%%%%%%%%%%%%%%%%%%%%%%%%%%%%%%%%%%%%%%
\begin{document}
  
  %% Titel
  \maketitle
  
  %%%%%%%%%%%%%%%%%%%%%%%%%%%%%%%%%%%%%%%%%%%%%%%%%%%%%%%%%%%%%%%%%%%%%%%%%%%%%%
  \section*{Problemstellung}
  %%%%%%%%%%%%%%%%%%%%%%%%%%%%%%%%%%%%%%%%%%%%%%%%%%%%%%%%%%%%%%%%%%%%%%%%%%%%%%
  
  In dieser Aufgabe haben wir das Konvergenzverhalten des $\theta$-Euler-Verfahren
  \begin{center}
  	$y^{i+1}=y^i+\tau((1-\theta)f(t_i,y^i)+{\theta}f(t_{i+1},y^{i+1})$ f\"ur $\theta \in [0,1]$, $i = 0,1,...$
  \end{center}
  untersucht. Dabei haben wir folgende Verfahren verwendet:
  \begin{itemize}
  	\item explizites Eulerverfahren $(\theta = 0)$
  	\item Crank-Nicolson-Verfahren $(\theta = 0.5)$
  	\item implizites Eulerverfahren $(\theta = 1)$
  \end{itemize}
  
  Wir wollen die experimentelle Konvergenzordnung (eoc = \underline{e}xperimental \underline{o}rder of \underline{c}onvergence) untersuchen. Als Test-Problem verwenden wir das Anfangswertproblem des \textit{Verhulst}-Modells zur Modellierung der Weltbev\"olkerung
  \begin{align*}
  	\tag{Verhulst}
  	y'=y(d-ay)
  \end{align*}
  mit Parametern $d=2.9*10^{-2}$ und $a=2.941*10^{-3}$.
  
  Die exakte L\"osung ist dabei gegeben durch
  \begin{align}
  	y(t)=\frac{d/a}{1-C exp(-d(t-t_0))}
  \end{align}
  mit $C\approx -2.25$ und $t_0=1960$ f\"ur $t\geq1960$.
  
  \newpage
  
  %%%%%%%%%%%%%%%%%%%%%%%%%%%%%%%%%%%%%%%%%%%%%%%%%%%%%%%%%%%%%%%%%%%%%%%%%%%%%%
  \section*{Ergebnis}
  %%%%%%%%%%%%%%%%%%%%%%%%%%%%%%%%%%%%%%%%%%%%%%%%%%%%%%%%%%%%%%%%%%%%%%%%%%%%%%

	\begin{figure}
\begin{tabular}{cc}
  \includegraphics[width=0.45\textwidth]{Bild1-1} &   \includegraphics[width=0.45\textwidth]{Bild1-2} \\
  \includegraphics[width=0.45\textwidth]{Bild1-3} &   \includegraphics[width=0.45\textwidth]{Bild1-4} \\
  \includegraphics[width=0.45\textwidth]{Bild1-5} &   \includegraphics[width=0.45\textwidth]{Bild1-6} \\
\end{tabular}
\caption{L\"osungsverlauf f\"ur unterschiedliche Schrittweiten}
\end{figure}

In Abbildung 1 sieht man den Verlauf der numerischen L\"osungen im Vergleich zur exakten L\"osung.

Wir speichern das Maximum \"uber den Fehler aller Zeitschritte
\begin{align*}
|E|_{\infty}:=\max_{i=1,...,N}|E_i|
\end{align*}
wobei
\begin{align*}
E_i:=y(t_i)-y^i
\end{align*}
der Fehler zwischen der Approximation $(y^i)_i$ und der exakten L\"osung $y$ ist.\\


F\"ur die Konvergenzordnung ben\"otigen wir die Absch\"atzung
\begin{align*}
|E|_{\infty}{\leq} C{\tau}^p
\end{align*}
wobei p die Konvergenzordnung ist, $\tau$ die Schrittweite und $C$ eine Konstante ist, die nur von dem Intervall $I = [0,T]$, der Lipschitz-Konstanten $L_f$ der rechten Seite der DGL und der zweiten Ableitung der exakten L\"osung abh\"angt.\\

Eine solche Absch\"atzung erh\"alt man durch das $eoctool$.
  
\end{document}
